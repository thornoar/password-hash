\documentclass[./main.tex]{subfiles}
%|return|[./main.tex]

\begin{document}

\section{Introduction}

The motivation behind the topic lies in the management of personal passwords. Nowadays, the average person requires tens of different passwords for different websites and services. Overall, one can distinguish between two ways of managing this set of passwords:

\begin{enumerate}
    \item \textbf{Keeping everything in one's head.} This is a method employed by many, yet it inevitably leads to certain risks. First of all, in order to fit the passwords in memory, one will probably make them similar to each other, or at least have them follow a simple pattern like "[shortened name of website]+[fixed phrase]". As a result, if even one password is guessed or leaked, it will be almost trivial to retrieve most of the others, following the pattern. Furthermore, the passwords themselves will tend to be memorable and connected to one's personal life, which will make them easier to guess. There is, after all, a limit to one's imagination.
    \item \textbf{Storing the passwords in a secure location.} Arguably, this is a better method, but there is a natural risk of this location being revealed, or of the passwords being lost, especially if they are stored physically on a piece of paper. Currently, various "password managers" are available, which are software programs that will create and store your passwords for you. It is usually unclear, however, how this software works and whether it can be trusted with one's potentially very sensitive passwords. After all, guessing the password to the password manager is enough to have all the other passwords exposed.
\end{enumerate}

In this paper I suggest a way of doing neither of these things. The user will not know the passwords or have any connection to them whatsoever, and at the same time the passwords will not be stored anywhere, physically or digitally. In this system, every password is a cryptographic hash produced by a fixed hashing algorithm. The algorithm requires two inputs: the public key, i.e. the name of the website or service, and the private key, which is an arbitrary positive integer known only to the user. Every time when retrieving a password, the user will use the keys to re-create it from scratch. Therefore, in order to be reliable, the algorithm must be "pure", i.e. must always return the same output given the same input. Additionally, the algorithm must be robust enough so that, even if a hacker had full access to it and its working, they would still not be able to guess the user's private key or the passwords that it produces. These considerations naturally lead to exploring pure mathematical functions as hashing algorithms and implementing them in a functional programming language such as Haskell.

\end{document}
